%%==================================================
%% conclusion.tex for BIT Master Thesis
%% modified by yang yating
%% version: 0.1
%% last update: Dec 25th, 2016
%%==================================================
\chapter{Conclusions and Policy Recommendations}
\section{Conclusions}


The culmination of this study into the effects of trade protectionism, particularly within the semiconductor industry, underscores the nuanced implications for carbon emissions on a global scale. The analytical results, as shown in Table \ref{tab:SummaryTable}, suggest that the current Sino-US trade, under the prevailing global production framework, contributes substantially to the reduction of global carbon footprint, an insight previously obscured by simplistic substitution methodologies.

The data reveals that the global carbon uplift percentage, which signifies the increase in the carbon footprint under the modelled scenarios, paints a telling picture of the benefits of current trade patterns. For instance, the no-trade carbon uplift globally for semiconductor-related products was marked at 0.9106\%, while the no Sino-US trade with gaming scenario exhibited a significantly higher uplift of 0.3568\% in China and 1.2301\% in the United States, signifying the substantial carbon mitigation benefits accrued through the current trade relations. This is contrasted with a global uplift of merely 0.0031\% under the no Sino-US trade scenario, which does not account for the more complex and realistic inter-country competitive dynamics.

Similarly, for upstream and downstream semiconductor products, the no-trade scenario indicates potential global uplifts of 0.5955\% and 0.3152\%, respectively. However, when incorporating a gaming trade redistribution approach, the results showcase an intriguing shift with a carbon uplift of -0.0043\% globally for upstream products, and a more pronounced uplift for downstream products at 0.1229\%. These variations underscore the intricate interplay between trade policies and environmental outcomes.

In conclusion, our comprehensive and granular approach to modeling trade restrictions and their impact on carbon emissions provides a more accurate representation of the potential environmental repercussions. The data-driven insights confirm that trade between China and the United States has been a key driver in mitigating carbon emissions, challenging the notion that protectionism could be a boon for environmental sustainability. As we navigate the complexities of global trade and its environmental impact, it becomes increasingly evident that collaboration, rather than protectionism, is pivotal in fostering a global reduction in carbon emissions.







\begin{table}
    \centering
    \caption{Summary of Carbon Uplift and Stressor Data for Semiconductivity Products}
    \label{tab:SummaryTable}
    \begin{tabular}{lccc}
    \hline
    \textbf{Products} & \textbf{Global} & \textbf{China} & \textbf{United States} \\
    \hline
    \multicolumn{4}{l}{\textit{Semiconductivity Related Products}} \\
    No Trade Carbon Uplift & 0.9106\% & -0.0768\% & -0.1678\% \\
    No Sino-US Trade Carbon Uplift & 0.0031\% & 0.0011\% & 0.0212\% \\
    No Sino-US Trade with Gaming Uplift & 0.3568\% & 1.2301\% & 0.0773\% \\
    Carbon Stressor (kg CO$_2$ eq./M.EUR) & 46,830.10 & 24,511.12 & 22,343.31 \\
    Net Export Rate & 0.0000\% & 25.4251\% & -26.6277\% \\
    \hline
    \multicolumn{4}{l}{\textit{Semiconductivity Upstream Products}} \\
    No Trade Carbon Uplift & 0.5955\% & -0.0280\% & -0.0590\% \\
    No Sino-US Trade Carbon Uplift & 0.0004\% & 0.0041\% & -0.0054\% \\
    No Sino-US Trade with Gaming Uplift & -0.0043\% & -0.0148\% & -0.0013\% \\
    Carbon Stressor (kg CO$_2$ eq./M.EUR) & 344,677.88 & 271,619.23 & 83,254.16 \\
    Net Export Rate & 0.0000\% & 21.3270\% & 65.2615\% \\
    \hline
    \multicolumn{4}{l}{\textit{Semiconductivity Downstream Products}} \\
    No Trade Carbon Uplift & 0.3152\% & -0.0488\% & -0.1088\% \\
    No Sino-US Trade Carbon Uplift & 0.0026\% & -0.0029\% & 0.0266\% \\
    No Sino-US Trade with Gaming Uplift & 0.1229\% & 0.3482\% & 0.1903\% \\
    Carbon Stressor (kg CO$_2$ eq./M.EUR) & 35,373.63 & 14,159.19 & 21,392.43 \\
    Net Export Rate & 0.0000\% & 25.5608\% & -27.0224\% \\
    \hline
    \end{tabular}
\end{table}

\section{Policy Recommendations}
The insights garnered from this study provide a pivotal foundation for formulating informed policy recommendations aimed at optimizing trade practices to enhance global environmental outcomes, particularly within the semiconductor industry. The evidence suggests that the interplay between international trade and carbon emissions is intricate and highly impactful. Therefore, it is essential for policy frameworks to leverage this relationship to foster a reduction in global carbon emissions through strategic trade agreements and regulations.

Firstly, the findings advocate for the continuation and enhancement of collaborative trade agreements, especially between major economies such as the United States and China. These partnerships have proven to be crucial in mitigating the carbon footprint associated with the semiconductor industry. Policies that encourage open trade, while also integrating strict environmental standards, can serve to both sustain economic growth and enhance environmental outcomes. Governments should consider trade policies that incentivize industries to adopt cleaner technologies and more efficient production methods. This could be achieved through tariff reductions on environmentally friendly goods and technologies, or through penalties for the trade of goods produced with high carbon footprints.

Secondly, the significance of technology transfer agreements in reducing global carbon emissions cannot be overstated. Developed countries, which often have access to more advanced technologies, should be encouraged to facilitate technology transfer to developing countries. This would not only help in leveling the playing field but also in reducing the global carbon emissions by enabling these countries to leapfrog to cleaner technologies. Such policies could be supported by international bodies and could include financial incentives or support mechanisms that make it easier for developing nations to access these technologies.

Additionally, there is a clear need for more comprehensive and transparent reporting and monitoring of carbon emissions on an international scale. The establishment of a global registry for carbon emissions from trade, which could track emissions from the production and transportation of goods, would enable better enforcement of international agreements and help in setting more targeted and effective carbon reduction goals.

Lastly, considering the complex dynamics revealed through the gaming trade redistribution scenarios, policies should also focus on the strategic distribution of production capabilities globally. Encouraging diversification of production and investment in renewable energy sources across different regions could help reduce the dependency on specific trade routes or production hubs, which often leads to higher emissions. Diversifying production can also help reduce vulnerabilities in the global supply chain, making it more resilient to shocks and stresses, which is increasingly important in a globalized economy.

In conclusion, it is imperative that trade policies are designed with a dual focus on economic growth and environmental sustainability. This research underscores the potential for trade to act as a lever for environmental improvement, provided it is managed and regulated thoughtfully and strategically. By embracing these policy recommendations, governments and international bodies can make significant strides towards achieving a sustainable balance between global trade efficiency and environmental conservation.