%%==================================================
%% abstract.tex for BIT Master Thesis
%% modified by yang yating
%% version: 0.1
%% last update: Dec 25th, 2016
%%==================================================
\begin{abstractEn}
    In the contemporary era of rapid technological advances and escalating trade protectionism, this thesis investigates the environmental implications of trade policies with a focus on the semiconductor industry. This research is crucial as it bridges the gap between global trade dynamics and environmental sustainability, highlighting the pressing need to understand and mitigate the environmental consequences of international commerce.

    This study employs a multi-regional input-output (MRIO) model to delve into the environmental impacts of trade, specifically analyzing how no-trade scenarios and a novel approach to global production share redistribution (GPSR, also refered as gaming scenario) influence carbon emissions between major trading nations like the United States and China. The focus on the semiconductor industry provides a detailed lens through which to assess both upstream and downstream effects on carbon footprints under various hypothetical trade restrictions. The methodology not only quantifies the direct impacts of trade but also explores sophisticated redistributive scenarios that reflect contemporary global trade complexities.

    The findings reveal that the current Sino-US trade relationship, particularly under the GPSR scenario, significantly mitigates global carbon emissions, with a documented carbon reduction of 0.3568\% globally for semiconductor related products. This contradicts common perceptions that trade protectionism inherently benefits environmental outcomes. Instead, the study demonstrates that strategic trade engagements and redistribution of production responsibilities can effectively reveal the carbon uplift under no trade scenario.

    Based on these insights, the thesis proposes robust policy recommendations aimed at fostering sustainable trade practices. These include enhancing international cooperation, integrating environmental standards into trade agreements, promoting technology transfers, and establishing comprehensive global emissions monitoring frameworks. Such policies are essential for leveraging trade as a tool for environmental improvement, ensuring that economic development does not come at the cost of ecological degradation.

    This research significantly contributes to academic discussions on trade and environment, offering policymakers and industry stakeholders valuable strategies to align economic activities with environmental conservation, thereby supporting a balanced approach to global trade and sustainability.

\end{abstractEn}

\begin{abstract}
    在快速技术进步和日益加剧的贸易保护主义的当今时代,本文研究了贸易政策对环境的影响,重点聚集于半导体行业。这项研究至关重要,因为它弥合了全球贸易动态与环境可持续性之间的差距,突显了理解和减轻国际商业环境后果的迫切需要。

本研究采用多区域投入产出(MRIO)模型深入探讨贸易的环境影响,特别是分析无贸易情景和全球生产份额重新分配(GPSR,也被后文指代为基于博弈分配的方法)这一新方法如何影响美国和中国等主要贸易国之间的碳排放。聚焦半导体行业提供了一个细致的视角,用以评估各种假设贸易限制下的上游和下游对碳足迹的影响。该方法不仅量化了贸易的直接影响,还探讨了反映当代全球贸易复杂性的复杂重新分配情景。

研究结果显示,当前中美贸易关系,特别是在GPSR情景假设下,显著减少了全球碳排放(半导体相关产品的全球碳减少量为0.3568\%)。这与普遍认为的贸易保护主义固有地有利于环境结果的看法相矛盾。相反,本研究表明,战略贸易参与和生产责任的重新分配可以有效地揭示无贸易情景下的碳上升。

基于这些洞见,论文提出了旨在促进可持续贸易实践的强有力政策建议。这些包括加强国际合作、将环境标准整合到贸易协议中、推广技术转移以及建立全面的全球碳排放监测框架。这些政策对于将贸易作为环境改善工具至关重要,确保经济发展不以生态退化为代价。

本研究对贸易与环境的学术讨论作出了重大贡献,为政策制定者和行业利益相关者提供了将经济活动与环境保护兼顾的宝贵策略,从而支持全球贸易和可持续性的平衡发展。
   

\end{abstract}
   

