%%==================================================
%% chapter01.tex for BIT Master Thesis
%% modified by yang yating
%% version: 0.1
%% last update: Dec 25th, 2016
%%==================================================
\chapter{Literature Review}
\label{chap:LITERATURE REVIEW}

\section{Environmental Consequences of Trade: Insights from Input-Output Analysis}
The environmental ramifications of global commerce have been meticulously scrutinized through the lens of input-output analysis, which has become a pivotal instrument for unraveling the complex relationship between economic activities and environmental impacts. Dating back to the 1970s, this methodology has provided significant insights (\supercitet{RN119}). In contemporary research, the scope of input-output analysis has broadened to include comprehensive assessments of environmental and societal repercussions within the entire span of global supply chains (\supercitet{RN158}; \supercitet{RN167}; \supercitet{RN171}; \supercitet{RN154}; \supercitet{RN172}). Pivotal examinations of global multi-regional input-output (GMRIO) databases, such as the ones articulated by \supercitet{RN170}, have been central to tracing the intricate connections linking nascent production zones to regions with high levels of consumption, shedding light on the extensive environmental and societal effects of international commerce. The progressive engagement of emerging economies in the global marketplace has precipitated an intensified migration of pollutive emissions to these locales, reaffirming the view that global trade amplifies environmental burdens in the developing world (\supercitet{RN127}; \supercitet{RN164}). Although there exists a temporal gap in the refreshment of input-output tables, sometimes extending to as late as 2014, these datasets remain crucial in characterizing socio-economic constructs and continue to hold significance for studies on trade. This scholarly landscape sets the stage for this investigation, which aims to delve deeper into the environmental outcomes of global trade, particularly under the influence of the recent shifts in the global economic climate.

\section{Trade Frictions and Environmental Outcomes: A Dualistic Evaluation}
Trade disputes are characterized by a multifaceted environmental impact, with both favorable and adverse consequences. On one flank, academic discourses, such as the one by \supercitet{RN141}, argue that trade constraints might lead to a tangible reduction in global carbon dioxide emissions by scaling back industrial operations and advocating for regional self-sustenance. Empirical evidence suggests that increased trade friction can mitigate greenhouse gas emissions, alter the global distribution of these emissions, and potentially enhance air quality across numerous countries. It postulates that continued governmental imposition of tariffs could precipitate a contraction in global greenhouse gas emissions, possibly up to 5\% (\supercitet{RN142}). Conversely, there are detrimental impacts to consider. Literature such as \supercitet{RN132} warns that while the dissolution of trade barriers can catalyze economic advancement and the adoption of greener technologies, the establishment of such barriers could obstruct these positive trends. In addition, the work of \supercitet{RN147} alongside \supercitet{RN142} emphasizes that trade disagreements, notably the trade conflict between China and the US, may instigate a surge in emissions in regions that shift to manufacturing hubs with more lenient environmental standards. These conflicting insights highlight the complexity inherent in the environmental effects of trade policies, underscoring the need for a holistic viewpoint to comprehend the intricate dynamic between trade tensions and global environmental health.

\section{Semiconductor Trade and Environmental Dynamics: A Critical Inquiry}
In the endeavor to contextualize the semiconductor industry within the broader framework of global trade and its environmental implications, one encounters a notable scarcity of comprehensive studies that intersect these three elements. Among the sparse literature that does delve into these intersections, the article by Mullen and Morris (2021)\cite{WOS:000657017300001} emerges as a crucial reference. They examine the ecological repercussions of semiconductor chip manufacturing from a life cycle perspective, emphasizing the pivotal role that nanofabrication technologies might play in enhancing sustainability within the sector .

Despite its insightful contributions, Mullen and Morris's study stands relatively alone in the academic landscape. It articulates concerns about the toxicological impacts associated with the vast quantities of solvents, acids, and gases used in chip production. Furthermore, it highlights the potential of directed self-assembly of block copolymers as a sustainable alternative to traditional lithographic processes. This perspective is invaluable, as it not only underscores the environmental challenges but also proposes innovative solutions within the semiconductor manufacturing process.

However, the current body of literature, lacking extensive examination of the interplay between semiconductor trade, environmental issues, and international commerce, does not reflect the complex dynamics illustrated in recent trade disputes, particularly those involving the United States and China. These disputes frequently revolve around the semiconductor industry, as detailed in Table \ref{tab:timeline}. For instance, legislative actions such as the U.S. CHIPS and Science Act of 2022, along with various sanctions and export restrictions, underscore the strategic importance and contentious nature of this sector in global trade relations.

This study aims to fill the noted gap by providing a thorough analysis of how semiconductor trade impacts environmental outcomes on a global scale, informed by these recent geopolitical developments. The unique and critical position of semiconductors in modern technology and international security, coupled with the environmental considerations highlighted by Mullen and Morris, justify a deeper exploration into this field. It is hoped that this research will catalyze further academic inquiry into the nuanced relationships between trade, technology, and the environment, particularly in sectors as pivotal as semiconductors.

\section{Global Production-Based Redistribution: A New Paradigm in Trade Impact Analysis}
Recent analyses of global trade dynamics have underscored the dual role of international commerce as both a vehicle for economic expansion and a catalyst for environmental change. A significant body of work (\supercitet{does_protectionism_improve_environment}; \supercitet{production_fragmentation_and_inter-provincial_trade_water_china}; \supercitet{China's_inter-provincial_trade}) has depicted the intricate relationship between the liberalization of trade and its environmental repercussions, particularly emphasizing the augmented burden of pollution borne by developing economies as a consequence of 'pollution outsourcing' by their developed counterparts. The advent of protectionist policies, resurging notably in the wake of the COVID-19 pandemic, presents an opportune research nexus to evaluate the environmental ramifications of such trade barriers.

Protectionism, though conventionally understood as an impediment to economic interdependence, introduces an intriguing dimension to environmental efficiency. The prevailing discourse (\supercitet{does_protectionism_improve_environment}) evaluates this through the lens of multi-regional input-output (MRIO) models and data envelopment analysis (DEA), presenting a dichotomy: while territorial emissions may wane under protectionist regimes, environmental efficiency does not necessarily parallel this decline. This divergence propels the discourse towards a more nuanced understanding of trade's environmental impact.

Inter-provincial analyses (\supercitet{production_fragmentation_and_inter-provincial_trade_water_china}) further delineate the multifaceted effects of trade, where production fragmentation and trade patterns reveal disparate impacts on regional water scarcity. It becomes evident that trade's influence extends beyond mere economic transfers, encompassing complex virtual water networks that underpin trade relationships.

In the context of China, trade has played a dichotomous role in advancing and inhibiting Sustainable Development Goals (SDGs), with inter-provincial trade being particularly pivotal (\supercitet{China's_inter-provincial_trade}). The examination of these SDG indicators through MRIO models over time furnishes insights into the spatial dynamics of trade's influence, demonstrating the heterogeneity of impacts across different provinces and timeframes.

The discourse then pivots to the quantified impacts of international trade on carbon intensity, where studies (\supercitet{impacts_of_US_carbon_intensity}) highlight the United States' progression towards emission reduction, facilitated to an extent by the transference of emissions through international trade. Such findings pivotally inform the current study, which innovates upon this dialogue by integrating a nascent perspective — the potential allocation of trade gaps based on a Nash equilibrium-inspired approach within the MRIO framework.

This study posits a groundbreaking methodology to estimate the repercussions of partial trade protectionism, specifically between China and the United States, on the semiconductor industry's global carbon footprint. Rejecting the oversimplified assumption of total domestic production replacement in the wake of trade cessation, this research proposes a redistribution mechanism predicated on the proportional global production shares detailed in the MRIO tables. This approach, enlightened by Nash equilibrium concepts, suggests that the competitive interplay following a trade gap would mirror pre-existing production and trade structures. Such an approach has not been previously explored in the literature, thereby charting a new trajectory for future research.

In essence, this literature review situates the current study at the vanguard of trade and environmental research. By transcending the conventional binaries of trade versus protectionism, it endeavors to capture the complexities of a globalized economy's adaptive mechanisms in the face of shifting trade policies.
